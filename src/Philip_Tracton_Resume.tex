%%%%%%%%%%%%%%%%%%%%%%%%%%%%%%%%%%%%%%%%%%%%%%%%%%%%%%%%%%%%%%%%%%%%%%%%%%%%%%%
%
% Resume for Philip Tracton
%
%%%%%%%%%%%%%%%%%%%%%%%%%%%%%%%%%%%%%%%%%%%%%%%%%%%%%%%%%%%%%%%%%%%%%%%%%%%%%%%
\documentclass[10pt,letter]{moderncv}

% moderncv themes
\moderncvtheme[blue]{classic}

% personal data
\firstname{Philip}
\familyname{Tracton}
\address{REMOVED FROM WEB}{Southern California}
\mobile{REMOVED FROM WEB}
\email{ptracton@gmail.com}
\extrainfo{\url{https://github.com/ptracton}}
% wanted

\usepackage[margin=0.5in]{geometry}

\begin{document}

\maketitle

\section{Experience}
\cventry{2010--Present}{Instructor}{UCLA Extension}{Westwood, CA}{}{
\begin{itemize}
    \item Took over teaching Learning Python.  This is an introduction to Python
    course that leads the students from installing Python to building their own weblog
    with a database backend. Taught class live and moved it to an online format.\\
    \item Took over teaching Embedded Software 1.   This class introduces
    students to programming in C for an embedded ARM Cortex-M3 microprocessor.  Students learn to write
    simple code to control hardware, handle interrupts and ultimately develop tasks
    running on FreeRTOS. \\
    \item Conceived and developed the Using FPGAs in Embedded Systems class.  The
    class introduces the students to Verilog and developing modules and test benches.
    Students ultimately program the Digilent Nexys-2 board with designs to control
    a variety of peripherals.
\end{itemize}
}

\cventry{2006--Present}{Sr. IC Design Engineer}{Medtronic
Neuromodulation}{Northridge, CA}{}{
\begin{itemize}
    \item NGMCU Inc 1 -- Installed and configured 2 IP blocks (DMA and SPI). 
    Implemented 2 custom blocks.  Hardware based firmware task scheduler and MAD
    to AHB Bridge.\\
    \item S0905a -- Lead firmware engineer for new ARM Cortex-M3 based System
    On Chip. Created CMSIS compliant proto-type device drivers, boot uCOS II,
    developed test applications, and boot ROM with symbolic linking all in
    simulation. Developed large demo application once silicon arrived.\\
    \item D452 -- Project lead for porting a pre-existing design from 0.6$\mu$
    AMI/ON to 0.25$\mu$ TSMC.  Replaced the memory and clock tree, added in ECC
    detection and control logic, significant power reduction while mantaining the
    same pad foot print.\\
    \item D281 -- Designed and implemented automated DSP blocks that could process
    data from $\Sigma\Delta$ ADCs and store in memory without CPU interaction
\end{itemize}
}

\cventry{2002--2006}{Embedded Software Engineer}{Medtronic
Diabetes}{Northridge, CA}{}{ \begin{itemize}
\item VGMS
\item IGMS
\item 2007 B/C/D
\end{itemize}
}

\cventry{2000--2002}{Firmware Engineer}{Zuma Networks}{Chatsworth, CA}{}{
Firmware for network box }

\cventry{1998--2000}{Embedded Software Engineer}{Teradyne}{Thousand Oaks, CA}{}{
Not much }

\section{Skills}
\cvline{Languages}{C, Python, Perl, Assembly (ColdFire, x86, PPC, Cortex-M3),
Verilog, VHDL, \LaTeX}
\cvline{Operating Systems}{Windows, Linux, FreeBSD, OpenBSD, FreeRTOS, uCOS}
\cvline{Tools}{Xilinx ISE, Altera Quartus, Modelsim}



\section{Education}
\cventry{2002--2005}{MS, Electrical Engineering}{California State University -
Nortridge}{Northridge, CA}{}{}
\cvline{Advisor:}{\small Professor Ramin Roosta}
\cvline{Project:}{\small The Dynamic Burnin of the forward and inverse 2D
Discrete Cosine Transform on a XC2V3000} 
\cventry{1993--1998}{BS, Electrical
Engineering}{University of Maryland}{College Park, MD}{}{}  % arguments 3 to 6 can be left empty

\end{document}
